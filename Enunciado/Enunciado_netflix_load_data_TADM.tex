% Options for packages loaded elsewhere
\PassOptionsToPackage{unicode}{hyperref}
\PassOptionsToPackage{hyphens}{url}
\PassOptionsToPackage{dvipsnames,svgnames*,x11names*}{xcolor}
%
\documentclass[
]{article}
\usepackage{lmodern}
\usepackage{amssymb,amsmath}
\usepackage{ifxetex,ifluatex}
\ifnum 0\ifxetex 1\fi\ifluatex 1\fi=0 % if pdftex
  \usepackage[T1]{fontenc}
  \usepackage[utf8]{inputenc}
  \usepackage{textcomp} % provide euro and other symbols
\else % if luatex or xetex
  \usepackage{unicode-math}
  \defaultfontfeatures{Scale=MatchLowercase}
  \defaultfontfeatures[\rmfamily]{Ligatures=TeX,Scale=1}
\fi
% Use upquote if available, for straight quotes in verbatim environments
\IfFileExists{upquote.sty}{\usepackage{upquote}}{}
\IfFileExists{microtype.sty}{% use microtype if available
  \usepackage[]{microtype}
  \UseMicrotypeSet[protrusion]{basicmath} % disable protrusion for tt fonts
}{}
\makeatletter
\@ifundefined{KOMAClassName}{% if non-KOMA class
  \IfFileExists{parskip.sty}{%
    \usepackage{parskip}
  }{% else
    \setlength{\parindent}{0pt}
    \setlength{\parskip}{6pt plus 2pt minus 1pt}}
}{% if KOMA class
  \KOMAoptions{parskip=half}}
\makeatother
\usepackage{xcolor}
\IfFileExists{xurl.sty}{\usepackage{xurl}}{} % add URL line breaks if available
\IfFileExists{bookmark.sty}{\usepackage{bookmark}}{\usepackage{hyperref}}
\hypersetup{
  pdftitle={Netflix enunciado práctica Netflix TADM 20\_21. MADM},
  pdfauthor={Grupo y nombre de cada usuario},
  colorlinks=true,
  linkcolor=red,
  filecolor=Maroon,
  citecolor=blue,
  urlcolor=blue,
  pdfcreator={LaTeX via pandoc}}
\urlstyle{same} % disable monospaced font for URLs
\usepackage[margin=1in]{geometry}
\usepackage{color}
\usepackage{fancyvrb}
\newcommand{\VerbBar}{|}
\newcommand{\VERB}{\Verb[commandchars=\\\{\}]}
\DefineVerbatimEnvironment{Highlighting}{Verbatim}{commandchars=\\\{\}}
% Add ',fontsize=\small' for more characters per line
\usepackage{framed}
\definecolor{shadecolor}{RGB}{248,248,248}
\newenvironment{Shaded}{\begin{snugshade}}{\end{snugshade}}
\newcommand{\AlertTok}[1]{\textcolor[rgb]{0.94,0.16,0.16}{#1}}
\newcommand{\AnnotationTok}[1]{\textcolor[rgb]{0.56,0.35,0.01}{\textbf{\textit{#1}}}}
\newcommand{\AttributeTok}[1]{\textcolor[rgb]{0.77,0.63,0.00}{#1}}
\newcommand{\BaseNTok}[1]{\textcolor[rgb]{0.00,0.00,0.81}{#1}}
\newcommand{\BuiltInTok}[1]{#1}
\newcommand{\CharTok}[1]{\textcolor[rgb]{0.31,0.60,0.02}{#1}}
\newcommand{\CommentTok}[1]{\textcolor[rgb]{0.56,0.35,0.01}{\textit{#1}}}
\newcommand{\CommentVarTok}[1]{\textcolor[rgb]{0.56,0.35,0.01}{\textbf{\textit{#1}}}}
\newcommand{\ConstantTok}[1]{\textcolor[rgb]{0.00,0.00,0.00}{#1}}
\newcommand{\ControlFlowTok}[1]{\textcolor[rgb]{0.13,0.29,0.53}{\textbf{#1}}}
\newcommand{\DataTypeTok}[1]{\textcolor[rgb]{0.13,0.29,0.53}{#1}}
\newcommand{\DecValTok}[1]{\textcolor[rgb]{0.00,0.00,0.81}{#1}}
\newcommand{\DocumentationTok}[1]{\textcolor[rgb]{0.56,0.35,0.01}{\textbf{\textit{#1}}}}
\newcommand{\ErrorTok}[1]{\textcolor[rgb]{0.64,0.00,0.00}{\textbf{#1}}}
\newcommand{\ExtensionTok}[1]{#1}
\newcommand{\FloatTok}[1]{\textcolor[rgb]{0.00,0.00,0.81}{#1}}
\newcommand{\FunctionTok}[1]{\textcolor[rgb]{0.00,0.00,0.00}{#1}}
\newcommand{\ImportTok}[1]{#1}
\newcommand{\InformationTok}[1]{\textcolor[rgb]{0.56,0.35,0.01}{\textbf{\textit{#1}}}}
\newcommand{\KeywordTok}[1]{\textcolor[rgb]{0.13,0.29,0.53}{\textbf{#1}}}
\newcommand{\NormalTok}[1]{#1}
\newcommand{\OperatorTok}[1]{\textcolor[rgb]{0.81,0.36,0.00}{\textbf{#1}}}
\newcommand{\OtherTok}[1]{\textcolor[rgb]{0.56,0.35,0.01}{#1}}
\newcommand{\PreprocessorTok}[1]{\textcolor[rgb]{0.56,0.35,0.01}{\textit{#1}}}
\newcommand{\RegionMarkerTok}[1]{#1}
\newcommand{\SpecialCharTok}[1]{\textcolor[rgb]{0.00,0.00,0.00}{#1}}
\newcommand{\SpecialStringTok}[1]{\textcolor[rgb]{0.31,0.60,0.02}{#1}}
\newcommand{\StringTok}[1]{\textcolor[rgb]{0.31,0.60,0.02}{#1}}
\newcommand{\VariableTok}[1]{\textcolor[rgb]{0.00,0.00,0.00}{#1}}
\newcommand{\VerbatimStringTok}[1]{\textcolor[rgb]{0.31,0.60,0.02}{#1}}
\newcommand{\WarningTok}[1]{\textcolor[rgb]{0.56,0.35,0.01}{\textbf{\textit{#1}}}}
\usepackage{graphicx,grffile}
\makeatletter
\def\maxwidth{\ifdim\Gin@nat@width>\linewidth\linewidth\else\Gin@nat@width\fi}
\def\maxheight{\ifdim\Gin@nat@height>\textheight\textheight\else\Gin@nat@height\fi}
\makeatother
% Scale images if necessary, so that they will not overflow the page
% margins by default, and it is still possible to overwrite the defaults
% using explicit options in \includegraphics[width, height, ...]{}
\setkeys{Gin}{width=\maxwidth,height=\maxheight,keepaspectratio}
% Set default figure placement to htbp
\makeatletter
\def\fps@figure{htbp}
\makeatother
\setlength{\emergencystretch}{3em} % prevent overfull lines
\providecommand{\tightlist}{%
  \setlength{\itemsep}{0pt}\setlength{\parskip}{0pt}}
\setcounter{secnumdepth}{5}
\renewcommand{\contentsname}{Contenidos}

\title{Netflix enunciado práctica Netflix TADM 20\_21. MADM}
\author{Grupo y nombre de cada usuario}
\date{2020}

\begin{document}
\maketitle

{
\hypersetup{linkcolor=blue}
\setcounter{tocdepth}{2}
\tableofcontents
}
\hypertarget{taller-evaluable-en-grupos-datos-netflix}{%
\section{Taller evaluable en grupos datos
netflix}\label{taller-evaluable-en-grupos-datos-netflix}}

Enlace a estos datos de
\href{https://www.kaggle.com/netflix-inc/netflix-prize-data}{Netflix}
Generad un proyecto nuevo. Bajad lo datos de netflix a un
carpeta/directorio que se llame \texttt{netflix} y dentro de
\texttt{netflix} crear una carpeta/directorio que se llame
\texttt{model\_netflix}.

Podéis (tenéis) que utilizar las ayudas del taller de estos datos.

\hypertarget{instrucciones}{%
\subsection{Instrucciones}\label{instrucciones}}

\begin{itemize}
\tightlist
\item
  Entregad en grupos de 2 ó 3 estudiantes.
\item
  Se puede hacer con R o python.
\item
  Hay que entregar el Rmd/notebook junto con su salida en html/pdf
\item
  Máxima longitud: 10 páginas en pdf.
\item
  Hay que cuidar la presentación, ortografía y redacción.
\item
  Fecha entrega 23 de diciembre.
\end{itemize}

\hypertarget{cuestiuxf3n-1-contexto-del-problema-y-modelo-de-datos-50}{%
\subsection{Cuestión 1: Contexto del problema y modelo de datos
(50\%)}\label{cuestiuxf3n-1-contexto-del-problema-y-modelo-de-datos-50}}

Como el problema es de datos masivos vamos cada grupo hará un muestreo
de los 4 ficheros. Para facilitar la labor os proporcionamos un fichero
en el que de cada película

\begin{Shaded}
\begin{Highlighting}[]
\KeywordTok{library}\NormalTok{(tidyverse)}
\end{Highlighting}
\end{Shaded}

\begin{verbatim}
## -- Attaching packages --------------------------------------- tidyverse 1.3.0 --
\end{verbatim}

\begin{verbatim}
## v ggplot2 3.3.2     v purrr   0.3.4
## v tibble  3.0.4     v dplyr   1.0.2
## v tidyr   1.1.2     v stringr 1.4.0
## v readr   1.4.0     v forcats 0.5.0
\end{verbatim}

\begin{verbatim}
## -- Conflicts ------------------------------------------ tidyverse_conflicts() --
## x dplyr::filter() masks stats::filter()
## x dplyr::lag()    masks stats::lag()
\end{verbatim}

\begin{Shaded}
\begin{Highlighting}[]
\KeywordTok{library}\NormalTok{(here)}
\end{Highlighting}
\end{Shaded}

\begin{verbatim}
## here() starts at C:/Developer MADM/Cursos MADM Github/proyecto-netflix-movies-madm
\end{verbatim}

\begin{Shaded}
\begin{Highlighting}[]
\NormalTok{filas_ID_combined_all=}\KeywordTok{read_csv}\NormalTok{(}\KeywordTok{here}\NormalTok{(}\StringTok{"Raw data"}\NormalTok{, }\StringTok{"filas_ID_combined_all.txt"}\NormalTok{))}
\end{Highlighting}
\end{Shaded}

\begin{verbatim}
## 
## -- Column specification --------------------------------------------------------
## cols(
##   X1 = col_character(),
##   fila = col_double(),
##   ID = col_double(),
##   fila_final = col_double(),
##   data = col_double()
## )
\end{verbatim}

\begin{Shaded}
\begin{Highlighting}[]
\KeywordTok{glimpse}\NormalTok{(filas_ID_combined_all)}
\end{Highlighting}
\end{Shaded}

\begin{verbatim}
## Rows: 17,770
## Columns: 5
## $ X1         <chr> "1:", "2:", "3:", "4:", "5:", "6:", "7:", "8:", "9:", "1...
## $ fila       <dbl> 1, 549, 695, 2708, 2851, 3992, 5012, 5106, 20017, 20113,...
## $ ID         <dbl> 1, 2, 3, 4, 5, 6, 7, 8, 9, 10, 11, 12, 13, 14, 15, 16, 1...
## $ fila_final <dbl> 548, 694, 2707, 2850, 3991, 5011, 5105, 20016, 20112, 20...
## $ data       <dbl> 1, 1, 1, 1, 1, 1, 1, 1, 1, 1, 1, 1, 1, 1, 1, 1, 1, 1, 1,...
\end{verbatim}

En total hay 17750 películas con ID entero de 1 a 17750

\begin{itemize}
\tightlist
\item
  La columna \texttt{X1} es de tipo character contiene el identificador
  original en el fichero \texttt{1:} un entero seguido de ``:''
\item
  La columna \texttt{fila} es de tipo integer contiene el número de fila
  que contiene el identificador de la película en el fichero
  \texttt{combinen\_data\_x.txt} el valor de x es viene determinado por
  la columna \texttt{file\_num}.
\item
  La columna \texttt{ID} es de tipo integer el identificado de la
  película sin \texttt{:}
\item
  La columna \texttt{fila\_final} es de tipo integer contiene el número
  de fila que contiene ella última entra de la película \texttt{ID}
\item
  La columna \texttt{file\_num} es de tipo integer contiene un entero de
  1 a 4 que indica si los datos de esa película están en el fichero
  \texttt{combinen\_data\_1.txt}, \texttt{combinen\_data\_2.txt},
  \texttt{combinen\_data\_3.txt} o \texttt{combinen\_data\_4.txt}
\end{itemize}

Cada fichero contiene una cierta cantidad de películas

\begin{Shaded}
\begin{Highlighting}[]
\KeywordTok{table}\NormalTok{(filas_ID_combined_all}\OperatorTok{$}\NormalTok{data)}
\end{Highlighting}
\end{Shaded}

\begin{verbatim}
## 
##    1    2    3    4 
## 4499 4711 4157 4403
\end{verbatim}

\begin{enumerate}
\def\labelenumi{\arabic{enumi}.}
\tightlist
\item
  Selecciona de las 1 a 17750 250 películas Semilla de grupo concatenar
  los dos últimos dígitos numéricos de vuestro DNI o tarjeta de
  residente
\end{enumerate}

\begin{Shaded}
\begin{Highlighting}[]
\CommentTok{# dos últimos dígitos 51 52 53 de cada miembro ordenados de menor a mayor 515253}
\CommentTok{#  y si hay ceros  segid este ejemplo}
\CommentTok{#  si las terminación del dni son 01 02 03 ordenadas de menor a mayor}

\KeywordTok{set.seed}\NormalTok{(}\DecValTok{01003}\NormalTok{)}
\KeywordTok{runif}\NormalTok{(}\DecValTok{4}\NormalTok{)}
\end{Highlighting}
\end{Shaded}

\begin{verbatim}
## [1] 0.4720480 0.9390508 0.1033403 0.8906890
\end{verbatim}

\begin{Shaded}
\begin{Highlighting}[]
\NormalTok{muestra_grupo=}\KeywordTok{sample}\NormalTok{(}\DecValTok{1}\OperatorTok{:}\DecValTok{12000}\NormalTok{,}\DecValTok{250}\NormalTok{,}\DataTypeTok{replace =} \OtherTok{FALSE}\NormalTok{)}
\KeywordTok{head}\NormalTok{(muestra_grupo)}
\end{Highlighting}
\end{Shaded}

\begin{verbatim}
## [1] 7897 5036 3874 2263 4340 9851
\end{verbatim}

Tenéis que localizar en el fichero \texttt{filas\_ID\_combined\_all}que
películas son en que fichero de \texttt{combined\_data\_?.txt} están y
las lineas que tenéis que leer.

\begin{enumerate}
\def\labelenumi{\arabic{enumi}.}
\item
  Contextualiza a partir de la información de Kaggle los datos de que
  disponemos. Qué datos contiene cada uno de los ficheros y para que´nos
  pueden resultar importantes para Netflix.
\item
  Leer cada película del fichero correspondiente y guardarlas,
  adecuadamente, en un mismo fichero para futuro tratamiento.
\item
  Construir el modelo de datos siguiendo las indicaciones de la taller
  ejemplo de netflix y generar la tibble netflix.
\item
  Leer el fichero de nombres y año y film que es
  \texttt{movie\_titles.csv} y hacer un \texttt{inner\_join} para
  disponer del título y año de estreno de cada película.
\item
  Guardar los datos procesado en un fichero csv, con el formato adecuado
  para utilizarlo en el siguiente apartado.
\end{enumerate}

\hypertarget{cuestiuxf3n-2-anuxe1lisis-exploratorio-eda.-50}{%
\subsection{Cuestión 2: Análisis exploratorio (EDA).
(50\%)}\label{cuestiuxf3n-2-anuxe1lisis-exploratorio-eda.-50}}

En las siguientes preguntas aplica todo lo que hemos visto acerca de la
documentación en el EDA: Título de gráficos, etiquetas de los ejes,
coloreado con información, leyendas, tablas bien presentadas
(knitr)\ldots{}

\begin{enumerate}
\def\labelenumi{\arabic{enumi}.}
\tightlist
\item
  Justifica para cada una de las variables de la tabla anterior el tipo
  de dato que mejor se ajusta a cada una de ellas: numérico, ordinal,
  categórico\ldots.
\item
  Estudia la distribución del numero de películas estrenadas por año.
  Realiza un gráfico de muestre esta distribución haciendo los ajustes
  necesarios (agrupaciones, cambios de escala, transformaciones\ldots)
\item
  Investiga la librería \texttt{lubridate} (o la que consideréis para
  manipulación de datos) y utilízala para transformar la columna de la
  fecha de la valoración en varias columnas por ejemplo
  \texttt{year},\texttt{month}, \texttt{week}, \texttt{day\_of\_week}.
\item
  Genera un tabla que para cada película nos dé el número total de
  valoraciones, la suma de las valoraciones, la media las valoraciones,
  y otras estadísticos de interés (desviación típica, moda , mediana).
\item
  De las cinco películas con más número total de valoraciones, compara
  sus estadísticos y distribuciones (histogramas, boxplot, violin
  plot,\ldots)
\item
  Investiga la distribución de valoraciones por día de la semana y por
  mes.¿Qué meses y días de la semana se valoran más películas en
  netflix?
\item
  Genera una tabla agrupada por película y año del número de
  valoraciones. Representa la tabla gráficamente para de las 10
  películas con mayor número de valoraciones .
\item
  Distribución del \texttt{score} promedio por año de las 10 películas
  con mayor número de valoraciones.
\item
  Realiza algún gráfico o estudió de estadísticos adicional que
  consideres informativo en base al análisis exploratorio anterior.
\end{enumerate}

\end{document}
